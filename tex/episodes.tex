\hypertarget{what-is-law}{%
\section{What is law?}\label{what-is-law}}

\hypertarget{what-do-we-mean-by-the-question-what-is-law}{%
\subsection{What do we mean by the question ``what is
law?''}\label{what-do-we-mean-by-the-question-what-is-law}}

\begin{itemize}
\tightlist
\item
  Is legal philosophy an attempt to understand the meaning of the term
  ``law''?

  \begin{itemize}
  \tightlist
  \item
    This is the semantic theory of law
  \item
    Not an interesting question. Why?

    \begin{itemize}
    \tightlist
    \item
      Law is an English word -- why should we care about a particular
      English word?
    \item
      Capacious word: applies to things we don't consider to be ``law''
      in the sense of legal rules

      \begin{itemize}
      \tightlist
      \item
        Think scientific laws, mathematics, grammar
      \end{itemize}
    \end{itemize}
  \end{itemize}
\item
  Instead, we want to know the nature of the thing that the word ``law''
  refers to: nature of legal institutions, rules, rights

  \begin{itemize}
  \tightlist
  \item
    Not a \emph{semantic} question, but a \emph{metaphysical} one
  \item
    Consider the question ``what is the nature of something?'' Two
    approaches:

    \begin{enumerate}
    \def\labelenumi{\arabic{enumi}.}
    \tightlist
    \item
      Could be interested in the identity -- the essential properties of
      an object
    \item
      Could be interested in the necessary properties
    \end{enumerate}
  \item
    Take the number three

    \begin{itemize}
    \tightlist
    \item
      Identity: number that comes after two
    \item
      Necessary properties: odd, divisor of nine

      \begin{itemize}
      \tightlist
      \item
        But these don't characterize its identity
      \end{itemize}
    \end{itemize}
  \item
    Legal philosophers want to know both identity and necessary
    properties of law
  \end{itemize}
\end{itemize}

\hypertarget{the-nature-of-legal-facts}{%
\subsection{The nature of legal facts}\label{the-nature-of-legal-facts}}

\begin{itemize}
\tightlist
\item
  What is a legal fact? Examples:

  \begin{itemize}
  \tightlist
  \item
    Bulgaria has a legal system
  \item
    In California, it is the law that you are not permitted to drive
    more than 65 miles per hour
  \end{itemize}
\item
  Legal facts are never metaphysically basic. Whenever a law or legal
  obligation exists, it always does so by virtue of some other fact(s)

  \begin{itemize}
  \tightlist
  \item
    C.f. laws of physics, which are metaphysically basic
  \item
    Legal facts depend upon their relationship to other, more basic
    facts
  \end{itemize}
\item
  What do legal facts depend on? Two possibilities:

  \begin{enumerate}
  \def\labelenumi{\arabic{enumi}.}
  \tightlist
  \item
    Social facts alone. People got together and decided something
  \item
    Also moral facts

    \begin{itemize}
    \tightlist
    \item
      Why is it against the law to drive more than 65 m.p.h.? Not simply
      that legislature and governor decided -- it's that people in CA
      \emph{ought} to listen to what they say

      \begin{itemize}
      \tightlist
      \item
        Perhaps because they are democratically elected
      \end{itemize}
    \end{itemize}
  \end{enumerate}

  \begin{itemize}
  \tightlist
  \item
    Whichever is true, this is a necessary property of law

    \begin{itemize}
    \tightlist
    \item
      Matters because if the first is true, then legal reasoning does
      not involve moral reasoning

      \begin{itemize}
      \tightlist
      \item
        Just need to figure out which social facts ground law
      \end{itemize}
    \item
      But if the second is true, need to use moral reasoning to figure
      out what the law is
    \end{itemize}
  \end{itemize}
\end{itemize}

\hypertarget{the-debate-between-legal-positivism-and-natural-law-theory}{%
\subsection{The debate between legal positivism and natural law
theory}\label{the-debate-between-legal-positivism-and-natural-law-theory}}

\begin{itemize}
\tightlist
\item
  Positivists: the law ultimately depends on social facts, never moral
  facts
\item
  Natural lawyers: law also depends on moral facts
\item
  John Austin: ultimately, law depends on power

  \begin{itemize}
  \tightlist
  \item
    Legal facts are legal facts because they have a certain relationship
    to brute political power, which is a social fact
  \end{itemize}
\item
  Ronald Dworkin: legal facts ultimately depend on moral facts

  \begin{itemize}
  \tightlist
  \item
    What makes the law a particular way is not just about what political
    actors have done, but about deep truths about moral philosophy
  \end{itemize}
\end{itemize}

\hypertarget{chicken-or-egg}{%
\section{Chicken or egg?}\label{chicken-or-egg}}

\hypertarget{the-possibility-puzzle}{%
\subsection{The possibility puzzle}\label{the-possibility-puzzle}}

\begin{itemize}
\tightlist
\item
  Chicken-egg puzzle: rules have come from some entity, but for that
  entity to have legal authority, there needs to be a rule giving them
  legal authority, but where did that come from?

  \begin{itemize}
  \tightlist
  \item
    Ex: Congress enacts stimulus plan
  \item
    Why is that legally valid? Article I of the Constitution
  \item
    Why is Article I legally valid? The Constitution has been ratified,
    and Article 7 says if ¾ of original states ratify the Constitution
    then it's valid
  \item
    But Article 7 is part of the Constitution. Seems circular. How can a
    provision that's part of a document confer legal authority on the
    document it's part of?
  \end{itemize}
\item
  Conflict between 2 principles:

  \begin{enumerate}
  \def\labelenumi{\arabic{enumi}.}
  \tightlist
  \item
    ``Egg principle'': some body has power to create legal norms only if
    existing norm confers that authority
  \item
    ``Chicken principle'': a legal rule exists only if someone with
    power to do so created it
  \end{enumerate}

  \begin{itemize}
  \tightlist
  \item
    End up in a vicious cycle or an infinite regress
  \end{itemize}
\item
  2 ways of solving:

  \begin{enumerate}
  \def\labelenumi{\arabic{enumi}.}
  \tightlist
  \item
    Deny the egg principle: certain bodies have authority even if there
    was no rule which conferred that authority

    \begin{itemize}
    \tightlist
    \item
      Austin does this: law is possible because there is an ultimate
      authority (the sovereign) who gets authority from pure political
      power
    \item
      Positivistic solution -- grounding legal facts in social facts
    \item
      But anyone who tries to solve the puzzle via social facts like
      this runs into Hume's puzzle
    \end{itemize}
  \item
    Deny chicken principle: legal rules can exist which confer authority
    even if nobody created those rules
  \end{enumerate}
\end{itemize}

\hypertarget{humes-puzzle}{%
\subsection{Hume's puzzle}\label{humes-puzzle}}

\begin{itemize}
\tightlist
\item
  Hume's law: can't derive an ought from an is

  \begin{itemize}
  \tightlist
  \item
    Normative conclusion has to be derived from a normative premise
  \end{itemize}
\item
  Positivism: legal facts don't depend on moral facts

  \begin{itemize}
  \tightlist
  \item
    Any kind of argument that establishes existence of legal facts will
    run into argument that's missing normative premise
  \end{itemize}
\item
  Austin: sovereign is someone who has brute political power

  \begin{itemize}
  \tightlist
  \item
    This is a positive statement, but the law seem to have normative
    claims
  \end{itemize}
\item
  Natural lawyers don't have this problem because they think are legal
  facts also grounded in moral facts, which are normative
\item
  If you don't believe in moral facts, do you have to be a legal
  positivist? Yes

  \begin{itemize}
  \tightlist
  \item
    If you don't believe in morality as objective, then the only way you
    can be a natural law theorist is if you deny that legal facts exist
  \item
    If you accept legal facts, then you can't be a natural law theorist
  \item
    Historically, positivists drawn to the theory because they are
    skeptical about existence of moral facts
  \end{itemize}
\item
  What if you think moral facts are constituted by social facts,
  i.e.~morality is a social construction? Does that mean you're a
  positivist? No

  \begin{itemize}
  \tightlist
  \item
    Social facts themselves are not metaphysically basic
  \item
    In jurisprudence, don't want to worry about methodological issues of
    what social/moral facts themselves depend on
  \end{itemize}
\end{itemize}

\hypertarget{john-austins-theory-of-law}{%
\subsection{John Austin's theory of
law}\label{john-austins-theory-of-law}}

\begin{itemize}
\tightlist
\item
  Law has two parts: theory of rules and theory of sovereignty. Law =
  rules + sovereignty
\item
  All rules are commands: expression of a wish, backed by a threat to
  inflict an evil if the wish is not fulfilled, issued by someone who is
  willing and able to act on the threat

  \begin{itemize}
  \tightlist
  \item
    Austin thought obligations were analytically related to commands in
    that people are under obligation when another expresses a wish,
    backed by threat, that they act or don't act a certain way
  \item
    Anytime someone commands someone to do something, they're obligated
    to do it
  \item
    All rules are ``general'' commands
  \end{itemize}
\item
  Who is the sovereign?

  \begin{itemize}
  \tightlist
  \item
    Obeyed by most people most of the time
  \item
    Must not habitually obey anybody else
  \end{itemize}
\item
  Law is general commands backed by threats issued by someone who is
  habitually obeyed and who habitually obeys nobody else

  \begin{itemize}
  \tightlist
  \item
    Can derive necessary properties: given what obligations are, all
    laws impose obligations
  \end{itemize}
\item
  Gives simple solution to the chicken-egg problem -- legal facts
  ``bottom out'' in sovereign

  \begin{itemize}
  \tightlist
  \item
    Whether you habitually obey someone is not a normative statement,
    it's a statement of social fact
  \end{itemize}
\item
  Also solves Hume's puzzle: obligations are not normative, they are
  descriptive

  \begin{itemize}
  \tightlist
  \item
    Any legal argument that has obligations as (descriptive) conclusions
    need not have normative premises
  \end{itemize}
\item
  Ultimate idea: law rests on power, habits of obedience, expresses of
  wishes backed by threats of sanctions

  \begin{itemize}
  \tightlist
  \item
    This theory dominant for a century (until Hart's ``Concept of Law'')
  \item
    But Austin is wrong!
  \end{itemize}
\end{itemize}

\hypertarget{harts-critique-of-austin-i}{%
\section{Hart's critique of Austin I}\label{harts-critique-of-austin-i}}

\hypertarget{power-conferring-rules}{%
\subsection{Power-conferring rules}\label{power-conferring-rules}}

\begin{itemize}
\tightlist
\item
  John Austin: law = rules + sovereignty

  \begin{itemize}
  \tightlist
  \item
    A rule is a general command

    \begin{itemize}
    \tightlist
    \item
      A command is the expression of wish issued by someone willing and
      able to impose an ``evil,'' called a sanction, if the wish is not
      obeyed
    \end{itemize}
  \item
    The sovereign is someone habitually obeyed by the bulk of population
    who habitually obeys nobody else
  \end{itemize}
\item
  Necessary properties:

  \begin{itemize}
  \tightlist
  \item
    A legal system exists when there is a sovereign who issues general
    commands
  \item
    We know the system's contents: what was commanded and what costs
    might be imposed if you don't listen

    \begin{itemize}
    \tightlist
    \item
      Look at \emph{behavioral regularities} between expressions of
      wishes, obedience, possibility of sanctions
    \end{itemize}
  \item
    Chicken came first: no need to find ultimate rules to convey
    authority on sovereign

    \begin{itemize}
    \tightlist
    \item
      Legal facts grounded in social facts (habitual obedience): a
      positivistic theory
    \end{itemize}
  \item
    Addresses Hume's puzzle: obligations, sovereignty, etc. are
    descriptive, not normative
  \end{itemize}
\item
  Hart critiques Austin's theory of rules

  \begin{itemize}
  \tightlist
  \item
    For Austin, all laws are ``duty-imposing''

    \begin{itemize}
    \tightlist
    \item
      Lots of legal rules are: duty not to kill other people, reasonable
      care, etc.
    \end{itemize}
  \item
    But there are other rules that ``confer power''

    \begin{itemize}
    \tightlist
    \item
      Constitutional law: set of laws that confer power
    \item
      Civil procedure: confer power on private citizens to bring suit,
      property rights
    \item
      We talk about these rules very differently
    \end{itemize}
  \item
    If you don't conform to a rule that imposes duty, you have
    ``violated'' it. You're ``guilty''

    \begin{itemize}
    \tightlist
    \item
      But we don't say that about power-conferring rules
    \item
      If you don't get two witnesses for a will, you're not ``guilty''
      -- you just didn't do what the law tells you to do if you want to
      have certain kinds of legal effects

      \begin{itemize}
      \tightlist
      \item
        No punishments or sanctions, but your actions are null and void:
        they have no legal effect
      \end{itemize}
    \end{itemize}
  \item
    Sometimes law is not trying to prevent you from doing something,
    it's trying to enable you to do something
  \end{itemize}
\end{itemize}

\hypertarget{nullities-as-sanctions}{%
\subsection{Nullities as sanctions}\label{nullities-as-sanctions}}

\begin{itemize}
\tightlist
\item
  Attempt to rehabilitate Austin

  \begin{itemize}
  \tightlist
  \item
    No sanctions associated with power-conferring rules?
  \item
    Austin's response: actually, power-conferring rules \emph{are}
    general commands backed by threat of sanction: nullity
  \item
    If you don't follow the rules, you are ``sanctioned'' in the sense
    that what you wanted to do won't be done
  \end{itemize}
\item
  Hart: but there is a difference between duty-imposing and
  power-conferring rules in their relationship to sanctions

  \begin{itemize}
  \tightlist
  \item
    Duty-imposing: sanction is logically detachable from rule itself

    \begin{itemize}
    \tightlist
    \item
      Can imagine a rule without a sanction
    \end{itemize}
  \item
    Power-conferring: can't do this

    \begin{itemize}
    \tightlist
    \item
      Can't have success without the possibility of failure
    \end{itemize}
  \item
    Sanction associated with duty-imposing rule is designed to make you
    \emph{worse off} than you would have been without sanction

    \begin{itemize}
    \tightlist
    \item
      Compare to nullity, which is not the imposition of cost, but
      refusal to confer a benefit
    \end{itemize}
  \end{itemize}
\end{itemize}

\hypertarget{kelsens-theory}{%
\subsection{Kelsen's theory}\label{kelsens-theory}}

\begin{itemize}
\tightlist
\item
  Hans Kelsen: agreed with Austin that law is composed of rules always
  backed by sanctions

  \begin{itemize}
  \tightlist
  \item
    Power-conferring rules are not real rules; they are fragments of
    duty-imposing rules which themselves impose sanctions
  \end{itemize}
\item
  Austin expanded category of sanctions to include nullities

  \begin{itemize}
  \tightlist
  \item
    Kelsen says no, sanctions are what Hart thinks they are, but
    power-conferring rules are not rules in their own right
  \end{itemize}
\item
  All rules have sanctions built-in

  \begin{itemize}
  \tightlist
  \item
    All rules are directed at legal officials, not ordinary citizens
  \item
    Direct them to impose sanctions under certain circumstances
  \item
    Ex: a testator has authority to distribute estate according to
    wishes if and only if they sign the will and two witnesses attest

    \begin{itemize}
    \tightlist
    \item
      Kelsen: that's only part of the rule. The rule is a long
      conditional where the antecedent is what the testator has to do,
      and the consequent is what duties the court is under regarding the
      executor carrying out the will
    \item
      Rule which seems to confer power on testator is really the
      conditions under which court is under duty to impose sanction on
      executor if they don't execute the will
    \end{itemize}
  \end{itemize}
\item
  Hart's response to Kelsen: this distorts the function of the law

  \begin{itemize}
  \tightlist
  \item
    Function of the law is to guide our conduct: guides testator
    conduct, guides executor's conduct, guides probate court's
    conduct\ldots{}
  \item
    Then it's silly to think all rules are directed at officials to
    impose sanctions on executors
  \item
    It's backwards to say that the wills law is directed at officials.
    It's directed at the people who want to write wills, the people who
    have to execute them, etc.
  \end{itemize}
\end{itemize}

\hypertarget{harts-critique-of-austin-ii}{%
\section{Hart's critique of Austin
II}\label{harts-critique-of-austin-ii}}

\hypertarget{duty-command-custom}{%
\subsection{Duty, command, custom}\label{duty-command-custom}}

\begin{itemize}
\tightlist
\item
  Austin: law is general commands backed by threats of sanctions issued
  by someone habitually obeyed and habitually obeys nobody else
\item
  Hart's critique: Austin says all laws are commands which impose
  duties, but this ignores power-conferring rules
\item
  Continue Hart's critique: even with respect to duty-imposing rules, is
  Austin's theory correct? Is it true that they really are general
  commands, backed by threats of sanctions?
\item
  People obey the law not just because they want to avoid sanctions, but
  because they think they're morally obligated to do so

  \begin{itemize}
  \tightlist
  \item
    Austin could say: sure, but what makes it \emph{the law} is that it
    is backed by sanctions
  \item
    But when we say the good citizen responds to the law because ``it's
    the law,'' what we mean is not just that they think law has content
    that is morally good

    \begin{itemize}
    \tightlist
    \item
      Consider the law against murder: the good citizen doesn't want to
      take life without justification and the fact that the law tells
      them not to murder doesn't add to their motivation
    \end{itemize}
  \item
    But there are laws that the good citizen responds to but not because
    their content is morally good, but simply because the law requires
    them

    \begin{itemize}
    \tightlist
    \item
      Whether I should pay my taxes at a certain rate is morally
      suspect, but the fact that the law requires me to do it means I
      should do it
    \end{itemize}
  \end{itemize}
\item
  Reason to obey the law is \emph{content-independent}

  \begin{itemize}
  \tightlist
  \item
    Can Austin's theory make sense of this?
  \item
    No.~For Austin, duty-imposing nature of the law is the fact that
    it's threatening an evil. But the good citizen is not following the
    law because of the evil threatened against them, but because the law
    demands it of them, and they think they are morally obligated to do
    what the law demands of them
  \item
    Austin gets duty-imposing nature of the law wrong by emphasizing
    sanctions
  \end{itemize}
\item
  Are laws commands? Sometimes. Police officer can tell you to stop,
  judge tells you to approach the bench, etc.

  \begin{itemize}
  \tightlist
  \item
    But most laws not imposed by command (created imperatively)
  \item
    Kelsen: in order to command something, you have to know what you're
    commanding. But legislation is created by legislators who don't know
    what's happening -- no legislator knows what content of an entire
    bill is
  \item
    Plus, laws apply to legislators themselves. Doesn't make sense to
    command oneself
  \item
    So Austin is wrong that laws are commands since laws are created in
    an impersonal, reflexive way
  \item
    Imperative theory of law can't explain the existence of customary
    law. Stare decisis (look to precedent) is customary but binding.
    Business customs can be legally binding in contract law

    \begin{itemize}
    \tightlist
    \item
      But custom is not command
    \end{itemize}
  \item
    Austin would say custom is not law until courts apply it. When they
    do so (and sovereign doesn't contradict them) then the sovereign has
    tacitly commanded people to act according to the custom

    \begin{itemize}
    \tightlist
    \item
      But courts aren't applying custom with the hope that sovereign
      will let them do it. They are doing so because it's custom, and
      custom is law
    \end{itemize}
  \end{itemize}
\end{itemize}

\hypertarget{sovereignty-as-habit}{%
\subsection{Sovereignty as habit}\label{sovereignty-as-habit}}

\begin{itemize}
\tightlist
\item
  Austin rejects the ``egg principle'': sovereignty depends on habits,
  not normative entities
\item
  Hart: legal sovereignty possesses a property that habits can't explain
  -- when one sovereign leaves and another takes their place, there's no
  break in legal continuity

  \begin{itemize}
  \tightlist
  \item
    Austin can't explain this, since a habit of obedience takes time to
    form
  \end{itemize}
\item
  Persistence of law: once a law is made, it sticks around until it's
  unmade, even when lawmaker is no longer there

  \begin{itemize}
  \tightlist
  \item
    But Austin says all laws are threats, and when person making the
    threats goes, the threat goes too
  \end{itemize}
\item
  For Austin, sovereign can't be limited by law, since they habitually
  obey nobody else

  \begin{itemize}
  \tightlist
  \item
    Idea of constitutionalism is hard to explain
  \item
    Who is the sovereign, according to Austin? Not a person -- it's a
    combination of institutions and individuals

    \begin{itemize}
    \tightlist
    \item
      In U.K., Austin thought king/queen in Parliament was sovereign
    \item
      In U.S., it's ``the people.'' How do the people obey themselves?
    \item
      Austin: they've commanded themselves wearing their ``sovereign''
      hat. When they obey the law they wear their ``subject'' hat
    \end{itemize}
  \end{itemize}
\item
  Introduces idea that sovereignty depends on rules (need to distinguish
  between sovereign/subject capacity with rules), not just habits
\end{itemize}

\hypertarget{internal-point-of-view}{%
\subsection{Internal point of view}\label{internal-point-of-view}}

\begin{itemize}
\tightlist
\item
  Hart wants to say Austin got it wrong by trying to rest all law on
  habits. Rather, the law rests on social \emph{rules}
\item
  Austin was right to think there is a component to social rules that
  involve behavioral regularities

  \begin{itemize}
  \tightlist
  \item
    But habits don't have a normative component
  \item
    When you're following a social rule, you think you \emph{should}
  \item
    Judge other people's conduct in this way as well
  \end{itemize}
\item
  Internal POV is not just the insider's POV, but a particular insider:
  somebody who's internalized the norms of their group

  \begin{itemize}
  \tightlist
  \item
    Take the group's standard as the standard that guides your conduct
    and the standard to judge others' conduct
  \end{itemize}
\item
  Social rule is a behavioral regularity and a critical, reflective
  attitude characterized by internal POV

  \begin{itemize}
  \tightlist
  \item
    You don't just act like everyone else does. You are guided by
    standard that you're taking as the way that you ought to act
  \end{itemize}
\item
  Hart says the internal POV is not necessarily the moral POV

  \begin{itemize}
  \tightlist
  \item
    Can take internal POV because you think it might advance your
    career, or because people will judge you if you don't
  \item
    Can accept a behavioral regularity as a standard for any reason, but
    when you've taken it as a standard, you've taken internal POV
  \item
    It's normative, but not a \emph{moral} statement
  \end{itemize}
\item
  Social rule is a social practice: behavioral regularity accepted from
  internal POV
\end{itemize}

\hypertarget{the-rule-of-recognition}{%
\section{The rule of recognition}\label{the-rule-of-recognition}}

\hypertarget{introduction}{%
\subsection{Introduction}\label{introduction}}

\begin{itemize}
\tightlist
\item
  Austin's theory: ignores existence of power-conferring rules, can't
  explain true nature of duty-imposing rules, can't explain continuity
  of legal authority, etc.
\item
  Can't explain intelligibility of law: we speak about law as
  institution that confers rights and imposes duties

  \begin{itemize}
  \tightlist
  \item
    Can't render normative nature of law intelligible because the theory
    is based on the non-normative, purely descriptive concept of power
  \item
    Power does not give you enough conceptual apparatus to explain
    rights, duties
  \item
    Category mistake: just because sovereign can force you to do
    something doesn't mean you \emph{ought} to do it
  \item
    Need to have law that gives us reasons to act: internal point of
    view
  \end{itemize}
\item
  Hart says we take law as a guide to conduct and a way to evaluate
  others' conduct

  \begin{itemize}
  \tightlist
  \item
    Social rule is not just behavioral regularity, it's accepted from
    internal POV
  \end{itemize}
\item
  At foundation of law: social rules/practices taken as guides of
  conduct, reasons for action
\item
  Imagine pre-legal society. All rules are customary: share food in a
  certain way, choose mates in a certain way, all accepted from internal
  POV

  \begin{itemize}
  \tightlist
  \item
    Problem of uncertainty: one part of group says act this way, other
    says different, no way to resolve this other than by counting heads
  \item
    Problem of stasis: no way to change the rules quickly, have to wait
    until people's behavior changes and accept it from internal POV

    \begin{itemize}
    \tightlist
    \item
      You want to be able to change behavior by changing rules. But if
      the rules are how people act, then the only way you can change
      rules is by changing behavior
    \end{itemize}
  \item
    Problem of inefficiency: no mechanism, institution to resolve
    disputes
  \end{itemize}
\item
  Law is a solution to these three problems

  \begin{itemize}
  \tightlist
  \item
    Uncertainty: rule of recognition is the rule that tells you which
    rules are binding
  \item
    Stasis: rule of change is the power-conferring rule that says who
    has ability to change the rules
  \item
    Inefficiency: rule of adjudication is the social rule that
    designates institution/person to decide whether rules have been
    obeyed
  \end{itemize}
\end{itemize}

\hypertarget{properties-of-the-rule-of-recognition}{%
\subsection{Properties of the rule of
recognition}\label{properties-of-the-rule-of-recognition}}

\begin{itemize}
\tightlist
\item
  Rule of recognition: identifies properties of rules that make it
  authoritative within group

  \begin{itemize}
  \tightlist
  \item
    In US: if Congress enacts legislation and signed by President (or
    overridden), then it's law, and judges have duty to apply/adjudicate
    it
  \item
    Secondary rule: a rule about other rules
  \item
    Social rule: practiced by group, accepted by internal POV
  \item
    Practiced only by officials. The rule of recognition is addressed to
    officials, like judges

    \begin{itemize}
    \tightlist
    \item
      Citizens do not accept the rule of recognition
    \end{itemize}
  \item
    Ultimate rule: its existence does not depend on any other rule
  \item
    Supreme rule: if anything conflicts, rule of recognition wins
  \item
    Tells you order of precedence of laws (ex: federal vs.~state)
  \item
    Duty-imposing rule: imposes duty on officials to apply certain rules
    to cases that arise before them
  \end{itemize}
\item
  Rule of change, adjudication are power-conferring
\item
  Constitution is not rule of recognition -- it's a collection of rules
  of change and adjudication
\item
  Resolve Austin's puzzles:

  \begin{itemize}
  \tightlist
  \item
    Continuity: Austin cannot explain continuity of legal authority
    because thinks all sovereignty is derived from habits

    \begin{itemize}
    \tightlist
    \item
      Hart: rule of recognition says whatever Rex II says becomes law
      when Rex I dies
    \item
      Continuous practice of rule recognition in the background
    \end{itemize}
  \item
    Persistence: Austin can't explain how rules made 100 years ago still
    bind, since the threat is no longer valid

    \begin{itemize}
    \tightlist
    \item
      Hart: rule of recognition says that if rule bears certain
      properties, then it is still a rule
    \end{itemize}
  \item
    Rule of recognition is a supreme rule, don't need idea of habitual
    obedience to a sovereign
  \item
    What makes a law a law? Validated by group's rule of recognition
  \item
    Can explain how rules apply to their makers

    \begin{itemize}
    \tightlist
    \item
      Legal rules are standards validated by rule of recognition, and
      you can make a standard that applies to yourself
    \end{itemize}
  \item
    Doesn't imply sovereign is above the law, because for Hart,
    sovereign is \emph{created} by the law
  \end{itemize}
\end{itemize}

\hypertarget{the-rule-of-recognition-rules}{%
\subsection{The rule of recognition
rules}\label{the-rule-of-recognition-rules}}

\begin{itemize}
\tightlist
\item
  Rule of recognition, change, adjudication are not validated by
  anything else. They are ultimate rules that exist by virtue of being
  practiced
\item
  Hart: two conditions for existence of legal system

  \begin{itemize}
  \tightlist
  \item
    Need secondary rules: rule of recognition, change, adjudication
  \item
    Officials have to accept these from internal POV

    \begin{itemize}
    \tightlist
    \item
      Non-officials don't have to accept these (not addressed to them)
    \item
      They have to follow the law most of the time, but not from
      internal POV

      \begin{itemize}
      \tightlist
      \item
        Can obey because afraid of being punished, or habit, doesn't
        matter
      \end{itemize}
    \end{itemize}
  \item
    Bifurcation: officials have to accept from internal POV, subjects
    have to obey law for any reason whatsoever
  \end{itemize}
\item
  Hart: essence of law is the union of primary and secondary rules

  \begin{itemize}
  \tightlist
  \item
    Primary rules are those validated by rule of recognition
  \end{itemize}
\end{itemize}

\hypertarget{critique-of-harts-theory}{%
\section{Critique of Hart's theory}\label{critique-of-harts-theory}}

\hypertarget{harts-solution-to-the-chicken-egg-puzzle-and-humes-challenge}{%
\subsection{Hart's solution to the chicken-egg puzzle and Hume's
challenge}\label{harts-solution-to-the-chicken-egg-puzzle-and-humes-challenge}}

\begin{itemize}
\tightlist
\item
  Austin's solution: chicken came first, i.e.~all law rests on
  sovereign, who is habitually obeyed and obeys nobody else. No further
  rule that makes sovereign the sovereign
\item
  Hart says that doesn't work. We need an egg: secondary social rules
  like the rule of recognition: a duty-imposing rule that sets out the
  criteria of legal validity
\item
  Who made secondary rules? Hart says nobody deliberately made them,
  rather, they are just social practices

  \begin{itemize}
  \tightlist
  \item
    Rule of recognition is created by legal officials listening to
    certain rules with certain characteristics, criticizing others when
    they don't
  \item
    Social rules are social practices
  \end{itemize}
\item
  Can a pattern of behavior impose duties? Let's say some judge doesn't
  follow the Constitution. All judges defer to Constitution and have
  critical reflective attitude towards that practice. That's a
  descriptive fact about the world -- where is the normative fact, that
  the judge \emph{ought} to follow the Constitution? This is Hume's
  challenge

  \begin{itemize}
  \tightlist
  \item
    How can you get the legal ``ought'' from the descriptive ``is'' of
    social practice?
  \item
    Hart opts for expressivism: there are no normative facts in the
    world

    \begin{itemize}
    \tightlist
    \item
      But we can take different attitudes towards descriptive facts -- a
      descriptive attitude or a normative attitude
    \end{itemize}
  \item
    Suppose someone has coronavirus. You can say ``this person has an
    infection and if you get near them, you could catch the virus''
    (descriptive attitude)

    \begin{itemize}
    \tightlist
    \item
      Or you can take normative attitude: ``Stay away from that person''
    \item
      Not an additional fact -- it's a different attitude to the fact
      that they are contagious
    \item
      Social practice can be engaged with either descriptively or
      normatively

      \begin{itemize}
      \tightlist
      \item
        You can take external POV or you can take internal POV -- take
        the social practice as a ``to-be-followedness''
      \end{itemize}
    \item
      Law: taking social practice of rule recognition as binding is
      deriving an ought from the general internal POV of approaching
      social practice from perspective of ``ought''-ness
    \end{itemize}
  \end{itemize}
\end{itemize}

\hypertarget{are-social-rules-social-practices}{%
\subsection{Are social rules social
practices?}\label{are-social-rules-social-practices}}

\begin{itemize}
\tightlist
\item
  Are social rules social practices? No

  \begin{itemize}
  \tightlist
  \item
    Seem like a category mistake: rules are abstract entities, but
    practices are concrete particulars

    \begin{itemize}
    \tightlist
    \item
      Rules have infinite domains, but practices are finite entities
    \end{itemize}
  \item
    Hart is trying to respond to Scandinavian realism: skeptical about
    rules

    \begin{itemize}
    \tightlist
    \item
      They wanted to say rules are just predictions about people would
      do
    \item
      Hart thinks this is a mistake -- rules have internal aspect. But
      Hart identified the rule with the practice
    \end{itemize}
  \item
    Maybe he should have said rules are \emph{created} by social
    practices

    \begin{itemize}
    \tightlist
    \item
      This doesn't work either
    \item
      Can have social practices without rules: don't take your laptop
      into the shower
    \item
      Some social practices give rise to rules. But we need to know what
      kinds of social practices give rise to social rules

      \begin{itemize}
      \tightlist
      \item
        Does the rule of recognition fall into this category?
      \end{itemize}
    \end{itemize}
  \end{itemize}
\item
  Perhaps the kind of social practices that give rise to social rules
  are \emph{coordination conventions}, which are recurring solutions to
  coordination problems

  \begin{itemize}
  \tightlist
  \item
    Practice/convention to drive on the right side leads to a social
    rule
  \item
    Everyone tips 20 percent leads to a social rule
  \item
    Perhaps following same rule of recognition is a coordination problem
    and gives rise to social rule

    \begin{itemize}
    \tightlist
    \item
      This is incorrect
    \item
      You follow a coordination convention because other people do so.
      Do officials in US follow Constitution because other officials do
      so?
    \item
      It's not arbitrary, like driving on right side of the road --
      officials have allegiance/respect for Constitution
    \end{itemize}
  \item
    Theory of law should not insist that judges are motivated for this
    particular reason
  \end{itemize}
\end{itemize}

\hypertarget{expressing-yourself-legally}{%
\subsection{Expressing yourself
legally}\label{expressing-yourself-legally}}

\begin{itemize}
\tightlist
\item
  Hart's solution to Hume's challenge: legal facts are just social facts
  but where we take a particular normative attitude (internal POV)

  \begin{itemize}
  \tightlist
  \item
    You can take internal POV for any reason, as long as you're
    committed to those social facts
  \item
    Causes problems:

    \begin{itemize}
    \tightlist
    \item
      Suppose bad man only follows the law because doesn't want to get
      punished. You don't have to be committed to law to figure out what
      law is
    \end{itemize}
  \item
    Hart's solution is not right -- can engage in legal reasoning
    without taking internal POV
  \item
    Can have selfish reasons for internal POV. Defendant says to judge:
    the only reason you're saying I broke the law is that you want to
    pick up your paycheck. How can I be punished because you just want
    to pick up your paycheck?

    \begin{itemize}
    \tightlist
    \item
      Judge criticizes defendants for reasons that don't apply to them
    \end{itemize}
  \end{itemize}
\item
  But Hart was right to say the law rests on rules, which rest on
  normative commitments
\end{itemize}

\hypertarget{the-hart-dworkin-debate}{%
\section{The Hart-Dworkin debate}\label{the-hart-dworkin-debate}}

\hypertarget{principles-in-hard-cases}{%
\subsection{Principles in hard cases}\label{principles-in-hard-cases}}

\begin{itemize}
\tightlist
\item
  Dworkin says Hart didn't answer ``what is law''

  \begin{itemize}
  \tightlist
  \item
    Critiques Hart's positivist view, which Dworkin says has 3 theses

    \begin{enumerate}
    \def\labelenumi{\arabic{enumi}.}
    \tightlist
    \item
      Every legal system has master test which distinguishes legal norms
      by pedigree (the way they were created rather than what they are)

      \begin{itemize}
      \tightlist
      \item
        Pedigree is the institutional source
      \item
        C.f. content, which asks, is it a good rule?
      \end{itemize}
    \item
      When rules run out, judges exercise discretion
    \item
      There are no legal obligations in the absence of legal rules
    \end{enumerate}
  \item
    Dworkin takes positivism to be a model of rules and argues law
    contains more than just rules
  \end{itemize}
\item
  1st thesis: Alludes to Hart's rule of recognition

  \begin{itemize}
  \tightlist
  \item
    Dworkin doesn't specify it must be a social rule, like Hart does
  \item
    Hart doesn't say anything about pedigree of rules, which Dworkin
    imputes
  \end{itemize}
\end{itemize}

\hypertarget{rules-vs.-principles}{%
\subsection{Rules vs.~principles}\label{rules-vs.-principles}}

\begin{itemize}
\tightlist
\item
  2nd thesis: seems trivially true
\item
  Dworkin uses specific characterization of a ``rule'': a norm that has
  an all-or-nothing character, c.f. a principle with ``dimension of
  weight''
\item
  Speed limit is a rule, applies in all-or-nothing manner
\item
  Principles can conflict, unlike rules
\item
  Dworkin says positivists think law is just rules, not principles
\item
  Weak discretion: exercise judgement in order to carry out an order

  \begin{itemize}
  \tightlist
  \item
    ``Do this at your discretion'' means your decision is not subject to
    review, but you're under an obligation to do \emph{something}
  \end{itemize}
\item
  Strong discretion: under no duty/obligation at all
\item
  Positivists say when rules run out, judges exercise strong discretion

  \begin{itemize}
  \tightlist
  \item
    No more rules, thus judges not under any duty to apply rules, thus
    they choose whatever standards they want
  \item
    Dworkin thinks this is a bad description of legal practice. Judges
    don't follow 2-step procedure: (1) check if rules apply (2) if not,
    make new law

    \begin{itemize}
    \tightlist
    \item
      They always act \emph{as if} there is law to apply
    \item
      Because law consists of principles as well as rules
    \end{itemize}
  \end{itemize}
\item
  In hard cases, rules run out but legal principles do not. Judges must
  weigh them against each other to figure out what to do

  \begin{itemize}
  \tightlist
  \item
    Exercising weak discretion
  \end{itemize}
\item
  Does Dworkin mischaracterize Hart?

  \begin{itemize}
  \tightlist
  \item
    Hart would have accepted that in addition to all-or-nothing
    standards, there are legal principles
  \item
    Dworkin trying to explain why Hart insisted judges exercise strong
    discretion
  \item
    Hart thought the law often runs out, so judges often exercise
    discretion and act like legislators
  \end{itemize}
\item
  If you're a positivist, you have to insist the law runs out since it
  depends on social facts alone

  \begin{itemize}
  \tightlist
  \item
    There is a limit to guiding people's conduct through social guidance
  \end{itemize}
\item
  Legal principles are legal because of their content, not because of
  their pedigree

  \begin{itemize}
  \tightlist
  \item
    Dworkin says this conflicts with first thesis
  \end{itemize}
\item
  In hard cases, judges look to principles

  \begin{itemize}
  \tightlist
  \item
    Thus the law doesn't consist of just rules (no strong discretion)
  \item
    Thus the rule of recognition can't identify principles as legal
    principles because legality derives from content, not pedigree
  \end{itemize}
\end{itemize}

\hypertarget{exclusive-vs.-inclusive-legal-positivism}{%
\subsection{Exclusive vs.~inclusive legal
positivism}\label{exclusive-vs.-inclusive-legal-positivism}}

\begin{itemize}
\tightlist
\item
  What can positivists say in response? Dworkin assumes all norms that
  courts under duty to apply are law

  \begin{itemize}
  \tightlist
  \item
    But can be under duty to apply norm that is not a \emph{legal} norm
  \item
    When court looks outside the law to morality, judge is not
    converting moral principles into the law
  \item
    When pedigreed rules run out, judges are under obligation to look to
    morality, but this doesn't make morality law
  \item
    Exclusive legal positivism: moral rules can never be legal rules
  \end{itemize}
\item
  Another response is to deny Dworkin's characterization: inclusive
  legal positivism

  \begin{itemize}
  \tightlist
  \item
    Rule of recognition says that when pedigreed standards run out,
    non-pedigreed standards win
  \item
    Accepts the idea that rule of recognition can incorporate
    non-pedigreed norms
  \end{itemize}
\item
  ELP: just because you're under obligation to apply moral principles,
  that doesn't convert moral principles to legal principles
\item
  ILP: rule of recognition can carry non-pedigreed standards. Law can be
  picked out by moral facts as long as social facts tell you to pick
  them out
\end{itemize}

\hypertarget{hart-on-interpretation}{%
\section{Hart on interpretation}\label{hart-on-interpretation}}

\hypertarget{formalism-and-open-texture}{%
\subsection{Formalism and open
texture}\label{formalism-and-open-texture}}

\begin{itemize}
\tightlist
\item
  Hart is trying to find middle ground between formalism and legal
  realism
\item
  What is formalism? Committed to 4 theses:

  \begin{enumerate}
  \def\labelenumi{\arabic{enumi}.}
  \tightlist
  \item
    Judges must always apply the law and can't override it
  \item
    Determinacy: for any question, law has an answer. There is always
    law to apply

    \begin{itemize}
    \tightlist
    \item
      Thus, law can't just be a list of rules, or it would have to be
      infinite
    \end{itemize}
  \item
    Conceptualism: In order to derive the law, one has to know a set of
    general principles from which it is possible to interpret concepts
    to derive the answer in a particular case

    \begin{itemize}
    \tightlist
    \item
      There are not an infinite number of rules, just a handful of
      general principles in any area of law
    \end{itemize}
  \item
    Amorality of adjudication: judges are not supposed to engage in
    moral reasoning
  \end{enumerate}
\item
  People think legal positivism (law rest on social facts alone, not
  moral facts) implies formalism

  \begin{itemize}
  \tightlist
  \item
    If law doesn't depend on moral facts, there are no moral facts to
    use when judges are deciding cases, so positivists are necessarily
    formalists; therefore positivism is wrong
  \end{itemize}
\item
  Hart says positivism is not committed to formalism
\item
  Why is formalism dumb? Judicial opinions are filled with moral
  arguments about ``what justice requires'' or ``virtuous behavior'' or
  ``obscenity''
\item
  Hart wants to argue positivism implies \emph{anti}-formalism

  \begin{itemize}
  \tightlist
  \item
    If you think there's always law to apply and judges are always
    supposed to apply it and never use moral reasoning, then it follows
    that law doesn't depend on moral facts
  \item
    But that's because you've assumed formalism is true (formalism →
    positivism)
  \end{itemize}
\item
  There are 2 ways to guide conduct, both incomplete:

  \begin{enumerate}
  \def\labelenumi{\arabic{enumi}.}
  \tightlist
  \item
    Precedent or authoritative example

    \begin{itemize}
    \tightlist
    \item
      Pointing to some person/action and saying ``do that''
    \end{itemize}
  \item
    General rule

    \begin{itemize}
    \tightlist
    \item
      Say ``men are supposed to take their hats off in church''
    \end{itemize}
  \end{enumerate}

  \begin{itemize}
  \tightlist
  \item
    In either case, there are gaps in guidance

    \begin{itemize}
    \tightlist
    \item
      Precedent: how similar to precedent does it have to be? Narrowly,
      broadly?
    \item
      General rule: what do you mean by ``hat,'' ``church,'' etc.?
    \end{itemize}
  \item
    Hart: General terms in natural language have ``open texture'' or
    vagueness

    \begin{itemize}
    \tightlist
    \item
      No way in natural language to get rid of this vagueness
    \item
      General terms have a ``core'' of settled examples and a
      ``penumbra,'' an area of vagueness
    \item
      Can't get around this by introducing more rules, since more
      natural language would have its own open texture
    \item
      What if you say ``when it's vague, decide in favor of defendant''?

      \begin{itemize}
      \tightlist
      \item
        But ``vagueness'' and ``open texture'' are themselves vague and
        open-textured, leading to higher-order vagueness
      \end{itemize}
    \end{itemize}
  \end{itemize}
\item
  ``Limits of the social'' argument: insofar as positivism believes that
  law is a set of standards that are picked out socially, law is going
  to be necessarily indeterminate because it can't cover every
  conceivable case

  \begin{itemize}
  \tightlist
  \item
    Because either you pick out law through authoritative example or
    you're using general terms in natural language
  \end{itemize}
\item
  Positivism implies anti-formalism: judges are going to have to rely on
  moral judgement to decide cases because they're going to run out of
  law to apply
\item
  Judges may act like they're finding the law, but when they apply
  morality to decide cases, they're making new law
\item
  Legal realism: reaction to formalism. Judges take principles and use
  them to make decisions based on policy considerations
\item
  Can't get from principles to a specific case without using normative
  reasoning, and it's disingenuous to pretend otherwise

  \begin{itemize}
  \tightlist
  \item
    When judges look at past cases, there just aren't enough cases to
    say ``this is what we should do''
  \end{itemize}
\end{itemize}

\hypertarget{rule-skepticism}{%
\subsection{Rule skepticism}\label{rule-skepticism}}

\begin{itemize}
\tightlist
\item
  Hart says you can't be a skeptic about primary rules

  \begin{itemize}
  \tightlist
  \item
    You can't say courts never follow the law, because there wouldn't be
    courts without the law that establishes the courts
  \end{itemize}
\item
  Mistake to point to open texture as proof that law is indeterminate

  \begin{itemize}
  \tightlist
  \item
    Yes, law has area of open texture, but also has area of settled
    meaning, a core
  \item
    So you also can't be a thoroughgoing skeptic about secondary rules
  \end{itemize}
\item
  Can be skeptic about complicated cases: Supreme Court decisions, cases
  in the penumbra -- no law to apply in these cases
\item
  Hart: people confuse finality with infallibility

  \begin{itemize}
  \tightlist
  \item
    Decision can be final but wrong
  \item
    E.g. impeachment: what is a high crime or misdemeanor? Whatever the
    Senate says it is? No.~It is true that nobody can challenge the
    Senate's decision about conviction. The decision is final, but it
    could be wrong
  \end{itemize}
\item
  Dworkin challenges the idea that when judges are looking to moral
  principles, they are making law

  \begin{itemize}
  \tightlist
  \item
    He thinks they're still \emph{finding} law, it's just that law
    depends on moral facts as well as social facts
  \end{itemize}
\item
  When it comes to resolving questions regarding the rule of
  recognition, judges make a decision, and if others accept it, it is
  the law going forward, because a new extension of social practice will
  have been formed

  \begin{itemize}
  \tightlist
  \item
    When it's accepted, it becomes part of the rule of recognition
  \item
    The fact of acceptance makes it correctly decided (retrospectively)
  \end{itemize}
\end{itemize}

\hypertarget{fullers-critique}{%
\subsection{Fuller's critique}\label{fullers-critique}}

\begin{itemize}
\tightlist
\item
  Judges are supposed to think about the purpose of the law
\item
  Because we think judges should approach legal interpretation by
  considering statute's purpose, it's a mistake to think law and
  morality are separate

  \begin{itemize}
  \tightlist
  \item
    The judge, in looking to purpose, is trying to figure out what you
    \emph{ought} to do
  \end{itemize}
\item
  Hart's positivism can't explain intuition that looking to purpose is
  important

  \begin{itemize}
  \tightlist
  \item
    This is wrong. It is possible to talk about purpose of statute
    without thinking about morality, e.g.~what were legislators trying
    to achieve in enacting a regulation? These are social facts
  \end{itemize}
\end{itemize}

\hypertarget{hart-on-law-and-morality}{%
\section{Hart on law and morality}\label{hart-on-law-and-morality}}

\hypertarget{justice-in-law}{%
\subsection{Justice in law}\label{justice-in-law}}

\begin{itemize}
\tightlist
\item
  Law is necessarily indeterminate. Social acts of guidance run out
  because

  \begin{enumerate}
  \def\labelenumi{\arabic{enumi}.}
  \tightlist
  \item
    For guidance via example/authoritative precedent, we don't know how
    close your case is to the precedent
  \item
    For guidance via general rules, natural language has open texture
  \end{enumerate}

  \begin{itemize}
  \tightlist
  \item
    Judges look to morality when law runs out
  \item
    But when they figure out what the law is, they look to the core; not
    the same as looking to morality
  \end{itemize}
\item
  For Hart, legal positivism means law and morality are not necessarily
  connected

  \begin{itemize}
  \tightlist
  \item
    Our definition of positivism is different: legal facts rest on
    social facts alone, not moral facts
  \item
    Law and morality are obviously connected in certain ways
  \end{itemize}
\item
  Law and justice have an important connection, seem to go together

  \begin{itemize}
  \tightlist
  \item
    Not everything that is morally bad is unjust, like tripping somebody
  \item
    Justice is maintaining or restoring balance of benefits and burdens

    \begin{itemize}
    \tightlist
    \item
      Lots of different types: distributive, retributive, etc.
    \item
      Formal justice: treating like cases alike
    \item
      Law seems to involve an element of formal justice

      \begin{itemize}
      \tightlist
      \item
        When you follow the law, you necessarily treat like cases alike
      \item
        Hart calls this ``justice in application''
      \end{itemize}
    \end{itemize}
  \item
    But there is a difference between justice in application and justice
    in the rules themselves

    \begin{itemize}
    \tightlist
    \item
      E.g. Jim Crow laws
    \end{itemize}
  \end{itemize}
\item
  Law and morality are similar, but have important differences

  \begin{itemize}
  \tightlist
  \item
    Similarities:

    \begin{itemize}
    \tightlist
    \item
      Legal and moral rules often coincide: ther are rules against
      assaulting people, engaging in fraud, destroying others'
      property\ldots{}

      \begin{itemize}
      \tightlist
      \item
        Morality tends to be more general (doesn't specify how fast you
        should drive)
      \end{itemize}
    \item
      Binding without consent: you don't agree to moral rules or to the
      law
    \item
      Constitute floors of acceptable behavior
    \item
      Supported by social pressure for conformity
    \end{itemize}
  \item
    Differences:

    \begin{itemize}
    \tightlist
    \item
      Moral rules are almost always important; legal rules can be
      trivial
    \item
      Morality immune to deliberate change: can't repeal morality
    \item
      Morality has an element of voluntariness: you can always say ``I
      couldn't help it''
    \item
      Moral rules are backed socially; law is backed by financial and
      physical sanctions (in addition to social pressure)
    \end{itemize}
  \end{itemize}
\end{itemize}

\hypertarget{minimum-content-of-the-natural-law}{%
\subsection{Minimum content of the natural
law}\label{minimum-content-of-the-natural-law}}

\begin{itemize}
\tightlist
\item
  Natural law typically seen as the opposite of legal positivism: argues
  there is a necessary connection between law and morality

  \begin{itemize}
  \tightlist
  \item
    Again, our definition is different: legal facts rest on moral facts
  \item
    ``Natural law'' sometimes taken to mean that morality is objective
    and discoverable by human reasoning
  \end{itemize}
\item
  Hart: doctrine of the minimum content of the natural law

  \begin{itemize}
  \tightlist
  \item
    Hart: law and morality are nomologically connected -- it's
    \emph{necessary} that human beings will have to have moral rules as
    part of their laws
  \item
    Law is geared towards social survival, hence there are rules to
    protect life, limb, and property
  \item
    5 features that necessitate these kinds of rules: We are\ldots{}

    \begin{enumerate}
    \def\labelenumi{\arabic{enumi}.}
    \tightlist
    \item
      Physically vulnerable to one another
    \item
      Roughly equal
    \item
      Altruistic to a limited degree
    \item
      Living in a world of scarcity
    \item
      Rational to limited degree
    \end{enumerate}
  \item
    Need rules to live together

    \begin{itemize}
    \tightlist
    \item
      Survival of group protected by rules, but that doesn't mean rules
      are just to everyone
    \end{itemize}
  \item
    Nomological necessity doesn't imply a morally just legal system
  \end{itemize}
\end{itemize}

\hypertarget{normative-legal-positivism}{%
\subsection{Normative legal
positivism}\label{normative-legal-positivism}}

\begin{itemize}
\tightlist
\item
  Every legal system requires not just coercive power, but acceptance of
  authority

  \begin{itemize}
  \tightlist
  \item
    2 claims: some have to take internal POV, and some have to accept
    legitimacy of authority
  \item
    First claim doesn't imply law and morality are connected

    \begin{itemize}
    \tightlist
    \item
      Some people have to take internal POV, but they can be small
      subset
    \item
      Can take internal POV for reasons that don't deal with morality
    \end{itemize}
  \item
    Second claim: Weber argued that any legal system must be based not
    just on coercion but on acceptance of authority by majority of
    population

    \begin{itemize}
    \tightlist
    \item
      This is again nomological but not metaphysical
    \end{itemize}
  \end{itemize}
\item
  Law contains many moral terms, but this doesn't mean law and morality
  are necessarily connected
\item
  When judges exercise moral discretion:

  \begin{itemize}
  \tightlist
  \item
    They can make mistakes, which leads law and morality to come apart
  \item
    They go beyond the law -- morality plays a role precisely because
    law has run out
  \end{itemize}
\item
  Seems as if our standards for evaluating legal systems use moral
  considerations

  \begin{itemize}
  \tightlist
  \item
    A ``good'' legal system is a \emph{morally} good legal system
  \item
    True that we evaluate law based on moral considerations, but law may
    fail to actually be good and it's still a legal system
  \end{itemize}
\item
  Law implies justice in application

  \begin{itemize}
  \tightlist
  \item
    But again, formal justice doesn't imply rules themselves are just
  \end{itemize}
\item
  Hart: If a law is unjust, you shouldn't obey it, so it seems as if law
  and morality are necessarily connected

  \begin{itemize}
  \tightlist
  \item
    True, but the law you're not supposed to obey is still the law
  \item
    Shouldn't confuse the idea that one is \emph{legally} obligated to
    follow the law, with the idea that one is \emph{morally} obligated
    to follow the law
  \item
    Natural lawyer says if law is not just, it's not law; positivist
    says it's still law, it's just law you shouldn't obey

    \begin{itemize}
    \tightlist
    \item
      How do you critique unjust law if, under the natural law view,
      it's not law?
    \end{itemize}
  \end{itemize}
\item
  ``Normative legal positivism'': the reason we should be legal
  positivists is that, morally speaking, we act better if we treat
  unjust laws as laws instead of denying that they're laws

  \begin{itemize}
  \tightlist
  \item
    Argument on moral grounds for being legal positivist
  \item
    But this is a bad way to argue for positivism

    \begin{itemize}
    \tightlist
    \item
      How do we know the best way to get people to act morally and
      disobey unjust rules?
    \end{itemize}
  \end{itemize}
\end{itemize}

\hypertarget{laws-empire-strikes-back}{%
\section{Law's Empire strikes back}\label{laws-empire-strikes-back}}

\hypertarget{recap}{%
\subsection{Recap}\label{recap}}

\begin{itemize}
\tightlist
\item
  View that positivism implies formalism

  \begin{itemize}
  \tightlist
  \item
    Formalists believe that judges never use moral reasoning
  \item
    Seemed that positivists are committed to this idea since they think
    law rests on social facts alone

    \begin{itemize}
    \tightlist
    \item
      So positivists would say judge should never rely on moral
      reasoning
    \end{itemize}
  \item
    Hart says positivism actually implies \emph{anti}-formalism

    \begin{itemize}
    \tightlist
    \item
      Social facts can't pick out comprehensive standards of guidance
      (``limits of the social'')
    \item
      Picking exemplars or using general terms always has areas of
      indeterminacy
    \item
      So judges inevitably have to rely on moral reasoning
    \end{itemize}
  \end{itemize}
\item
  Dworkin says judges \emph{never} act as though the law is
  indeterminate

  \begin{itemize}
  \tightlist
  \item
    Always act as though there is law to find by looking to principles
    that are valid because they are morally appropriate
  \item
    Not as though the law is indeterminate and judges make new law
  \item
    So formalism is true and therefore positivism must be false
  \end{itemize}
\item
  Inclusive legal positivism: judges do look to morality, but social
  facts say that moral facts are relevant

  \begin{itemize}
  \tightlist
  \item
    Rule of recognition says that when socially designated standards run
    out, they should look to morality
  \end{itemize}
\item
  Exclusive legal positivism: the fact that judges look outside the law
  does not make these extra-legal standards the law
\end{itemize}

\hypertarget{dworkins-critique-of-legal-positivism}{%
\subsection{Dworkin's critique of legal
positivism}\label{dworkins-critique-of-legal-positivism}}

\begin{itemize}
\tightlist
\item
  Dworkin shows why the legal positivist response won't work using a new
  critique: ``the problem of theoretical disagreements''

  \begin{itemize}
  \tightlist
  \item
    Most powerful objection to positivism
  \end{itemize}
\item
  Dworkin distinguishes \emph{propositions of law:} statement of a legal
  fact in a particular jurisdiction, which can be true or false by
  virtue of the \emph{grounds of law}: the facts that render a
  proposition of law true

  \begin{itemize}
  \tightlist
  \item
    Ex: Propositions of law in California are true if a bill that
    expresses the proposition is approved by a majority of the
    legislature, signed by governor, etc.
  \item
    In Hart's terminology, grounds of law are part of the criteria of
    legal validity
  \end{itemize}
\item
  Dworkin says people can disagree about the grounds of law in 2 ways:

  \begin{enumerate}
  \def\labelenumi{\arabic{enumi}.}
  \tightlist
  \item
    Empirical disagreement about whether grounds of law are satisfied in
    a given case
  \item
    Theoretical disagreement about what the grounds of law \emph{are} in
    a particular legal system
  \end{enumerate}
\item
  Dworkin says according to positivism, all disagreements about
  propositions of law are really empirical, never theoretical

  \begin{itemize}
  \tightlist
  \item
    But it is common for legal actors to have theoretical disagreements
    and positivism can't explain this
  \end{itemize}
\item
  Dworkin says positivists accept ``plain-fact'' view of the law:

  \begin{itemize}
  \tightlist
  \item
    Grounds of law are the facts that people agree render propositions
    in that jurisdiction true
  \item
    Grounds of law must refer to matters of historical fact/pedigree

    \begin{itemize}
    \tightlist
    \item
      This is saying all positivists are exclusive legal positivists
    \item
      But most people are inclusive legal positivists
    \end{itemize}
  \end{itemize}
\item
  Dworkin says there are disagreements about grounds of law: which facts
  you look to in order to determine whether a proposition is true or
  false

  \begin{itemize}
  \tightlist
  \item
    Positivists can't explain this because for them because for them
    grounds of law are determined by consensus
  \item
    If there's disagreement, then something cannot be a ground of law
  \item
    For positivists, disagreement about grounds of law leads to
    incoherence
  \end{itemize}
\end{itemize}

\hypertarget{constructive-interpretation}{%
\subsection{Constructive
interpretation}\label{constructive-interpretation}}

\begin{itemize}
\tightlist
\item
  No way out of this by being an exclusive or inclusive legal
  positivist: for both, the grounds of law are determined by consensus

  \begin{itemize}
  \tightlist
  \item
    Difference between them is on what kinds of things there can be
    consensus on

    \begin{itemize}
    \tightlist
    \item
      Inclusive legal positivist says consensus on moral facts is still
      law
    \end{itemize}
  \end{itemize}
\item
  Counterargument: positivism is correct but legal actors are incoherent

  \begin{itemize}
  \tightlist
  \item
    But this is not a small disagreement, this is a fundamental feature
    of law: it is determined by consensus among legal actors
  \end{itemize}
\item
  Another counterargument: legal actors argue for things they don't
  believe in, even if they involve conceptual mistakes

  \begin{itemize}
  \tightlist
  \item
    But it's not the fact of the theoretical disagreement itself that is
    insincere, it's that they're arguing for a specific position
  \end{itemize}
\item
  Analogy to literary criticism: theoretical disagreements about what
  Shakespeare meant, not empirical disagreements about what he said

  \begin{itemize}
  \tightlist
  \item
    At bottom, having a disagreement about what makes literature
    valuable and what kind of methodology would make it the most
    valuable it could be
  \item
    Constructive interpretation: ``imposing purpose on an object or
    practice in order to make it the best possible example of the form
    or genre to which it is taken to belong''
  \item
    Ask ``what makes literature great?'' and then pick a methodology
    that reflects that
  \item
    In law: imposing moral theory about what makes law valuable, grounds
    of law are the ones that make law morally the best that it can be

    \begin{itemize}
    \tightlist
    \item
      How do you make legal practice the best it can be? The
      interpretation that best \emph{fits} and \emph{justifies} legal
      practice
    \end{itemize}
  \end{itemize}
\item
  Fit: extent to which it approves of the object's existence or its
  properties

  \begin{itemize}
  \tightlist
  \item
    One purpose fits better than another when it recommends behavior
    that more closely matches observed conduct
  \item
    Ex: 2 interpretations of Christmas: rank commercialism vs.~goodwill
    and peace on earth

    \begin{itemize}
    \tightlist
    \item
      Former fits better because practice of Christmas is more in line
      with rank commercialism than peace on earth
    \end{itemize}
  \end{itemize}
\item
  Justification: which of the purposes is more morally justified

  \begin{itemize}
  \tightlist
  \item
    Goodwill and peace on earth is more justified than rank
    commercialism
  \end{itemize}
\item
  How to balance the two? Maybe justification is more important:
  Christmas is better when we see it as goodwill and peace on earth
\end{itemize}

\hypertarget{law-as-integrity}{%
\section{Law as integrity}\label{law-as-integrity}}

\hypertarget{concept-vs.-conception}{%
\subsection{Concept vs.~conception}\label{concept-vs.-conception}}

\begin{itemize}
\tightlist
\item
  Positivists say law ultimately depends on social facts
\item
  Dworkin says no, morality is at the foundation of law -- judges look
  to morality in hard cases

  \begin{itemize}
  \tightlist
  \item
    Positivists say rule of recognition has moral principles as criteria
    for validity
  \item
    Exclusive positivists: judges using moral principles doesn't make
    them into legal principles
  \end{itemize}
\item
  Both exclusive and inclusive legal positivists accept the idea that
  criteria for legal validity (grounds of law) are determined by
  consensus

  \begin{itemize}
  \tightlist
  \item
    Dworkin: that assumes there \emph{is} consensus, but there are
    theoretical disagreements about what the grounds of law are
  \item
    Positivists can't explain the possibility of disagreements
  \end{itemize}
\item
  Constructive interpretation: imposing purpose on social practice so it
  can be the best possible example of the thing it is

  \begin{itemize}
  \tightlist
  \item
    Of the law: make legal practice the best it can be, morally
  \item
    Need fit and justification
  \end{itemize}
\item
  Which grounds of law fit and justify legal practice?
\item
  For Dworkin, 2 stages to legal interpretation:

  \begin{enumerate}
  \def\labelenumi{\arabic{enumi}.}
  \tightlist
  \item
    What are the grounds of law?
  \item
    In a given case, do the grounds of law apply?

    \begin{itemize}
    \tightlist
    \item
      In order to figure out which propositions are true and decide the
      case
    \end{itemize}
  \end{enumerate}
\item
  Dworkin approaches the first stage in two substages: figure out
  concept of law, then the right conception of law

  \begin{itemize}
  \tightlist
  \item
    Concept: what would make the law the best it can be at very abstract
    level
  \item
    Conception: more precise characterization

    \begin{itemize}
    \tightlist
    \item
      3 different conceptions, ask which one makes the law the best it
      can be
    \end{itemize}
  \end{itemize}
\item
  Concept of law: law is about ensuring that the state does not use
  coercion unless it is compatible with the rights and responsibilities
  that flow from past political acts

  \begin{itemize}
  \tightlist
  \item
    Dworkin thinks everyone will sign onto this, but maybe this doesn't
    hold up
  \item
    Law doesn't just \emph{limit} the state, but \emph{guides} the state
  \end{itemize}
\end{itemize}

\hypertarget{conventionalism}{%
\subsection{Conventionalism}\label{conventionalism}}

\begin{itemize}
\tightlist
\item
  3 conceptions that purport to make law the best it can be

  \begin{enumerate}
  \def\labelenumi{\arabic{enumi}.}
  \tightlist
  \item
    Conventionalism
  \item
    Pragmatism
  \item
    Law as integrity
  \end{enumerate}
\item
  Conception: interpretation of a particular legal practice
  (e.g.~American law), purports to identify 2 things:

  \begin{itemize}
  \tightlist
  \item
    The moral purpose of the practice
  \item
    The sets of facts such a purpose commends as the grounds of law
  \end{itemize}
\item
  A conception of law should fit and justify the practice

  \begin{itemize}
  \tightlist
  \item
    Fits: facts it identifies are actually the grounds of law recognized
    by officials
  \item
    Justifies: purpose it assigns to law is morally justifiable
  \end{itemize}
\item
  Conventionalism: rights and responsibilities flow from past political
  acts when they are specifically decided by those acts

  \begin{itemize}
  \tightlist
  \item
    To identify rights and responsibilities, look to the conventions of
    the system
  \item
    Purpose: to give people fair warning and protect their expectations
    about how the state is going to act
  \item
    Grounds: those given by convention
  \item
    If you respect convention, then you will be protecting people's
    expectations and giving them fair warning because everyone knows
    what the conventions are
  \item
    Sounds like exclusive legal positivism, but it isn't

    \begin{itemize}
    \tightlist
    \item
      Positivism: convention determines what the law is, but not because
      this makes the law the best it can be
    \item
      Conventionalism is the ``Dworkinian spin'' on positivism
    \end{itemize}
  \item
    When conventions run out, act in the way you think is best

    \begin{itemize}
    \tightlist
    \item
      Gives judges flexibility
    \end{itemize}
  \item
    Does it fit?

    \begin{itemize}
    \tightlist
    \item
      No.~Even when conventions run out, people still disagree about
      what the grounds of law are, and judges still feel constrained by
      the law
    \end{itemize}
  \item
    Is it justified?

    \begin{itemize}
    \tightlist
    \item
      No.~Balance between fair warning and flexibility doesn't track
      cases where there's convention or no convention

      \begin{itemize}
      \tightlist
      \item
        Sometimes good to have flexibility even when conventions are
        clear
      \item
        Sometimes good to protect expectations even when there are no
        conventions
      \end{itemize}
    \end{itemize}
  \end{itemize}
\item
  Pragmatism: Richard Posner's theory of interpretation -- rejects the
  idea that the past constrains the present

  \begin{itemize}
  \tightlist
  \item
    Judge should make the decision that maximizes community welfare
  \item
    Skeptical conception of law: what legal officials did in the past
    doesn't matter
  \item
    This doesn't fit the way judges think of past political acts, nor
    does it seem justified
  \end{itemize}
\item
  Law as integrity: rights and responsibilities flow from past political
  acts when they are conferred and imposed by principles and policies
  that portray these political acts in their best light

  \begin{itemize}
  \tightlist
  \item
    If we want to know what makes the law the best it can be, law as
    integrity says: grounds of law should be determined by the
    principles and policies that make the law the best it can be
  \item
    The purpose that is furthered by those grounds of law is the ideal
    of integrity

    \begin{itemize}
    \tightlist
    \item
      Integrity: acting according to the set of principles and policies
      that are consistent across all members of the community in all
      cases
    \item
      ``Requires governments to speak with one voice, act in principled
      and coherent manner towards all its citizens''
    \end{itemize}
  \item
    If we treat some members of a community some way, even if we treated
    them that way mistakenly, ideal of integrity requires that we treat
    other members that way too
  \end{itemize}
\end{itemize}

\hypertarget{hercules}{%
\subsection{Hercules}\label{hercules}}

\begin{itemize}
\tightlist
\item
  Rawls's theory of justice: justice as fairness
\item
  Dworkin: like the best interpretation of justice is fairness, the best
  intepretation of law is integrity
\item
  Grounds of law are those that present past political acts in their
  best light
\item
  Does it fit?

  \begin{itemize}
  \tightlist
  \item
    Dworkin says yes, because it can explain something that other
    conceptions of law can't: our abhorrence of ``checkerboard
    statutes''

    \begin{itemize}
    \tightlist
    \item
      Checkerboard statute makes arbitrary distinctions on matters of
      important principles
    \end{itemize}
  \item
    This is an unfair comparison. Conventionalism fits some aspects of
    legal practice just like law as integrity fits some aspects of legal
    practice
  \item
    You could say law as integrity doesn't fit because of federalism,
    which is a kind of checkerboard solution
  \end{itemize}
\item
  Is it justified?

  \begin{itemize}
  \tightlist
  \item
    Wny is state justified in imposing coercion? Not because we
    consented, not because it's fair, but because we're part of
    ``fraternal community'' where we live according to same set of
    principles
  \item
    Generates associative obligations to obey the law
  \item
    To investigate this, you need to do serious political philosophy
  \end{itemize}
\item
  If you accept law as integrity, how do you intepret the law? This is
  the second stage

  \begin{itemize}
  \tightlist
  \item
    Dworkin says look at principles that will put past political acts in
    their best light

    \begin{itemize}
    \tightlist
    \item
      Need to take into account \emph{all} past political acts -- need
      to be superhuman
    \item
      Dworkin calls his judge ``Hercules''
    \item
      Thinks of the judge as a moral-political philosopher; legal issues
      always pose philosophical issues
    \end{itemize}
  \end{itemize}
\end{itemize}

\hypertarget{why-i-am-not-an-inclusive-legal-positivist-part-i}{%
\section{Why I am not an inclusive legal positivist, part
I}\label{why-i-am-not-an-inclusive-legal-positivist-part-i}}

\hypertarget{raz-on-authority}{%
\subsection{Raz on authority}\label{raz-on-authority}}

\begin{itemize}
\tightlist
\item
  Possible content of rule of recognition

  \begin{itemize}
  \tightlist
  \item
    Inclusive legal positivism: criteria of legal validity can be
    anything, as long as legal officials accept them
  \item
    Exclusive legal positivism: law has certain constraints, so a norm
    is not a legal norm just because it's morally appropriate -- need
    some institutional source
  \end{itemize}
\item
  Most positivists are inclusive legal positivists

  \begin{itemize}
  \tightlist
  \item
    Seems like it's descriptively superior: constitutional provisions of
    many legal systems formulated in moral terms (e.g.~``cruel and
    unusual punishment'', ``human dignity shall be inviolable'')
  \item
    Easiest way to respond to Dworkin: judges look to principles in hard
    cases because rule of recognition requires them to do so
  \end{itemize}
\item
  Joseph Raz's argument against inclusive legal positivism

  \begin{itemize}
  \tightlist
  \item
    A necessary feature of all legal systems is that they claim
    justified practical authority over a population

    \begin{itemize}
    \tightlist
    \item
      Claim moral authority to impose obligations on citizens to obey
      its directives
    \end{itemize}
  \item
    If that claim is to be intelligible, legal system must be capable of
    exercising authority
  \item
    When does the law have legitimate authority?

    \begin{itemize}
    \tightlist
    \item
      When legal officials are believed in their claim to authority by
      some part of the population -- they are \emph{de facto}
      authorities
    \end{itemize}
  \item
    Service conception of authority: legal systems have legitimate
    authority when they provide a service
  \item
    Each of us constantly faced with the question ``what should I do?''
  \item
    We balance the ``first-order reasons for action'' and choose the
    best-supported action
  \item
    Sometimes rationality suggests that we should not attempt to engage
    in balancing of first-order reasons; sometimes we should ignore the
    reasons we have

    \begin{itemize}
    \tightlist
    \item
      Why? If doing so would enable us to act on balance of first-order
      reasons better than if we complied with them directly
    \item
      Second-order reasons to exclude first-order reasons

      \begin{itemize}
      \tightlist
      \item
        E.g. you make a plan in advance because you know in the moment
        you'll mess up the balancing of first-order reasons
      \end{itemize}
    \end{itemize}
  \item
    Raz says legitimate authorities provide us with second-order,
    exclusionary reasons
  \item
    Normal justification thesis: normally, authorities give us
    directives so that, by following them, we better conform to reasons
    we have independent of those directives than if we tried to conform
    to those reasons directly
  \end{itemize}
\end{itemize}

\hypertarget{the-argument-from-authority}{%
\subsection{The argument from
authority}\label{the-argument-from-authority}}

\begin{itemize}
\tightlist
\item
  When is it possible to have legitimate authority? When directives are
  second-order exclusionary reasons that help you fulfill first-order
  reasons. 2 cases:

  \begin{enumerate}
  \def\labelenumi{\arabic{enumi}.}
  \tightlist
  \item
    Coordination: by following directives of authority, can contribute
    to realization of public goods
  \item
    Expertise: authorities know how the first-order reasons apply to us
    better than we do (public health, national security, environmental
    protection)
  \end{enumerate}
\item
  Necessary that interpretation of authoritative directives can't depend
  on first-order reasons they are intended to exclude and replace

  \begin{itemize}
  \tightlist
  \item
    But this is what inclusive legal positivism allows subjects of
    authority to do
  \item
    Inclusive legal positivism says criteria of legal validity could be
    moral fairness of a norm
  \item
    But the whole point of a legal directive is to replace those reasons
    and exclude them from your deliberation

    \begin{itemize}
    \tightlist
    \item
      Fairness is a first-order reason for action -- law is supposed to
      consider these and issue a directive that excludes and replaces
      that first-order reason
    \item
      But the inclusive legal positivist says you can consider those
      first-order reasons when you decide whether that directive is
      authoritative
    \item
      That undoes the very idea of having the authoritative directive
    \end{itemize}
  \item
    If in order to know what directive is you have to answer the
    question the authority is claiming to answer, the purpose of having
    legitimate authority evaporates
  \end{itemize}
\item
  2 problems with this:

  \begin{enumerate}
  \def\labelenumi{\arabic{enumi}.}
  \tightlist
  \item
    Links two things a positivist wouldn't have linked: conceptual
    question of what law is with the moral question of when it's
    legitimate

    \begin{itemize}
    \tightlist
    \item
      Conceptual: all law claims legitimate authority
    \item
      Moral: theory of when law generates moral obligation to obey
    \item
      Linking the two: in order to know what the law is conceptually,
      you have to know when it's justified

      \begin{itemize}
      \tightlist
      \item
        Weird for positivist to say that what the law is depends on the
        political philosophy of when it's justified
      \item
        Sounds much more like Dworkin
      \end{itemize}
    \end{itemize}
  \item
    To claim legitimate authority, you must have it

    \begin{itemize}
    \tightlist
    \item
      Makes anarchism an incoherent theory because anarchism says law
      claims authority but doesn't have it
    \end{itemize}
  \end{enumerate}
\end{itemize}

\hypertarget{the-practical-difference-thesis}{%
\subsection{The practical difference
thesis}\label{the-practical-difference-thesis}}

\begin{itemize}
\tightlist
\item
  Inclusive legal positivism: certain legal rules can't guide conduct,
  only \emph{pedigreed} rules can
\item
  Positivists think function of law is to guide conduct

  \begin{itemize}
  \tightlist
  \item
    Norms cannot guide conduct
  \item
    Law must make a practical difference to our reasoning
  \end{itemize}
\item
  In order for a rule to guide conduct, it has to motivate you to act in
  a way differently than if you hadn't appealed to the law
\item
  Can't say law is guiding conduct if appealing to it never makes a
  difference in how you act or how you're motivated to act
\item
  Moral principles that lack institutional sources can't make practical
  differences
\item
  If inclusive rule of recognition makes a practical difference, the
  moral principle validated by it can't make a difference

  \begin{itemize}
  \tightlist
  \item
    Rule of recognition tells judge to follow morality, so regardless of
    whether they follow the actual moral principle, they do the same
    thing
  \end{itemize}
\item
  Argument doesn't work for exclusive rule of recognition

  \begin{itemize}
  \tightlist
  \item
    Law passed does make a difference: if law is repealed, judge might
    act differently
  \item
    Doesn't work for moral principles because these can't be repealed
  \end{itemize}
\end{itemize}

\hypertarget{planning-theory-of-law-i}{%
\section{Planning theory of law I}\label{planning-theory-of-law-i}}

\hypertarget{planning}{%
\subsection{Planning}\label{planning}}

\begin{itemize}
\tightlist
\item
  Conclude neither positivist theory (Austin, Hart) are successful
\item
  Dworkin's alternative: legal facts ultimately depend on moral facts as
  well

  \begin{itemize}
  \tightlist
  \item
    Possibility of theoretical disagreements
  \end{itemize}
\item
  Alternative positivistic theory that tries to integrate law with other
  phenomena in non-legal sphere -- activity of planning

  \begin{itemize}
  \tightlist
  \item
    One area where everyone is a positivist: plans
  \item
    If we plan to take the train to the beach, morality has nothing to
    do with it
  \item
    We can think of legal activity as social planning
  \end{itemize}
\item
  Starting point: ``legal activity'' is what legal officials,
  legislatures, courts, administrative agencies do

  \begin{itemize}
  \tightlist
  \item
    Creation, application, enforcement of legal norms
  \item
    Legal activity is activity of social planning

    \begin{itemize}
    \tightlist
    \item
      Deciding actions we may or may not do and which people have the
      authority to decide what we may or may not do
    \end{itemize}
  \end{itemize}
\item
  What is planning? Michael Bratman says what makes human beings unique
  is not just that we're rational (means-ends calculations) but that we
  have the ability to plan

  \begin{itemize}
  \tightlist
  \item
    Incremental: we fill in our plans over time
  \item
    Past decisions form a framework that guides new decisions
  \item
    Hart: we don't want all legal rules to answer every question, want
    to leave open texture for judges to decide
  \item
    This is the way planning works: leave degree of flexibility to fill
    plan in as future becomes the present
  \item
    Legislation also has this ``fill in as you go along'' structure --
    it becomes more specific, determinate over time
  \end{itemize}
\end{itemize}

\hypertarget{law-as-a-planning-organization}{%
\subsection{Law as a planning
organization}\label{law-as-a-planning-organization}}

\begin{itemize}
\tightlist
\item
  Legal activity is not just the activity of social planning, but also a
  \emph{shared} activity

  \begin{itemize}
  \tightlist
  \item
    Something that legal officials do together
  \item
    Legal officials are following shared plan, which tells them what
    each of their roles are
  \item
    Hart's rule of recognition is part of this shared plan, part of the
    way legal officials coordinate
  \end{itemize}
\item
  Legal activity is \emph{official} -- those who engage in social
  planning inhabit offices, stable/permanent positions of authority
  where turnover of occupants is possible and expected
\item
  Legal activity is \emph{institutional} -- legal relations obtain
  independent of intentions of people involved

  \begin{itemize}
  \tightlist
  \item
    If shared plan validates a proposed plan (e.g.~a majority of
    legislators vote yes), then the law is legally valid regardless of
    why the legislators voted that way
  \item
    Laws are not commands because you can enact laws without knowing
    what they are

    \begin{itemize}
    \tightlist
    \item
      Social plans can be created as long as people follow the shared
      plan -- they may not intend to create the plan they are creating
    \end{itemize}
  \end{itemize}
\item
  Legal activity forms an \emph{organization}

  \begin{itemize}
  \tightlist
  \item
    Parents are not legal systems: they are shared planners, but they
    don't form an organization
  \item
    Their planning is not institutional (you have to \emph{know} what
    you're planning for your kids)
  \end{itemize}
\item
  Legal activity is a \emph{compulsory} planning organization -- you
  can't say you didn't consent to a particular law
\end{itemize}

\hypertarget{the-moral-aim-thesis}{%
\subsection{The moral aim thesis}\label{the-moral-aim-thesis}}

\begin{itemize}
\tightlist
\item
  What kind of compulsory planning organization is the law?
\item
  If you think of law as an organization, participants engaged in
  collective activity because they accept a shared plan, then it's easy
  to see why legal positivism makes sense: organizations are created
  through intentions/actions of groups (social facts)

  \begin{itemize}
  \tightlist
  \item
    Morality seems to play no role
  \item
    Positivism becomes an obvious position to take
  \end{itemize}
\item
  But purpose of legal organization is moral in nature

  \begin{itemize}
  \tightlist
  \item
    Fundamental aim of law is to rectify moral deficiency associated
    with circumstances of legality
  \item
    Meet moral demand of the circumstances of legality in efficient
    manner

    \begin{itemize}
    \tightlist
    \item
      Enable communities to solve numerous, serious problems that would
      otherwise be too costly/risky to resolve
    \end{itemize}
  \item
    Doesn't give up on positivism: law has a moral aim, but it doesn't
    have to satisfy that moral aim in order to be law

    \begin{itemize}
    \tightlist
    \item
      Constitutive aim of an assertion is to tell the truth, but it's
      still an assertion if the person is lying
    \item
      Legal systems that don't solve moral problems they're supposed to
      solve are still legal systems, they're just bad/unjust
    \end{itemize}
  \item
    Motivations/justifications

    \begin{enumerate}
    \def\labelenumi{\arabic{enumi}.}
    \tightlist
    \item
      Modern life needs law because we have complicated moral problems
    \item
      Legal systems unable to solve serious moral problems are
      criticizable
    \item
      Consider criminal organizations. They are institutional,
      compulsory, but they are not legal systems because we don't think
      of them as aiming to solve a moral problem -- they are part of the
      moral problem of the circumstances of legality

      \begin{itemize}
      \tightlist
      \item
        When organized crime does something good, we think of it as
        serendipitous, but not so with the law
      \item
        We think the law is \emph{supposed} to solve moral problems,
        unlike criminal organizations
      \end{itemize}
    \end{enumerate}
  \end{itemize}
\item
  So far, what is law: compulsory planning organization with a moral aim

  \begin{itemize}
  \tightlist
  \item
    But not all compulsory planning organization with a moral aim are
    legal systems
  \item
    Consider a condo board: creates rules for residents, applies them in
    cases of dispute. Constituted by officials whose intentions are
    irrelevant, rules binding irrespective of consent. Aim to solve
    problems that can't be solved through other forms of social
    ordering. But not a legal system. Why not?
  \item
    Maybe: legal systems always claim \emph{supreme authority}, and
    Florida law trumps the condo rules

    \begin{itemize}
    \tightlist
    \item
      This argument doesn't work because then Florida wouldn't have a
      legal system either, since federal law supersedes Florida law
    \end{itemize}
  \end{itemize}
\item
  Although state law must comply with federal law, federal law
  automatically presumes state law complies with federal law

  \begin{itemize}
  \tightlist
  \item
    Federal law gives states the benefit of the doubt, allows them to
    make and enforce law without first checking
  \item
    By contrast, states afford nothing comparable to private actors
  \item
    Not enough for rules made by private actors to be in compliance with
    state law. They are not allowed to enforce laws unless private actor
    demonstrates compliance with state laws
  \item
    State police can yank skinny dipper from pool. Condo board can't do
    that -- they need to call police, prosecutor needs to be involved

    \begin{itemize}
    \tightlist
    \item
      Condo board doesn't enjoy same presumption of validity/compliance
      as state government
    \end{itemize}
  \end{itemize}
\item
  Legal systems are self-certifying systems

  \begin{itemize}
  \tightlist
  \item
    Planning organization is self-certifying when it enjoys
    \emph{presumption of validity} from superior planning organization
  \item
    Thus, condo board is not a legal system, but Florida law is
  \end{itemize}
\end{itemize}

\hypertarget{planning-theory-of-law-ii}{%
\section{Planning theory of law II}\label{planning-theory-of-law-ii}}

\hypertarget{solving-the-puzzles}{%
\subsection{Solving the puzzles}\label{solving-the-puzzles}}

\begin{itemize}
\tightlist
\item
  Possibility puzzle: legal authority needs norm that confers power, but
  need someone with legal authority to confer that power

  \begin{itemize}
  \tightlist
  \item
    Planning theory follows ``egg'' principle: all authority comes from
    norms of instrumental rationality
  \item
    We are planning creatures, able to coordinate across time and
    people, and through this form of partial-staged deliberation we
    solve problems we couldn't otherwise solve
  \item
    The foundation of legal system is a shared plan -- most officials
    accept their part in the plan
  \item
    Ability to adopt and share a plan is unmysterious
  \item
    Planning theory rests legal authority on power of planning agents to
    adopt norms to coordinate activities over time and across persons
    (instrumental rationality)
  \end{itemize}
\item
  Austin: all laws backed by sanctions, ultimately rest on power

  \begin{itemize}
  \tightlist
  \item
    Plans can be duty-imposing or power-conferring
  \item
    No problem with continuity/persistence (sovereign dies, power passed
    on because shared plan says new person is sovereign)
  \end{itemize}
\item
  Hart: assimilated social rules of recognition, change, and
  adjudication to behavior of legal officials

  \begin{itemize}
  \tightlist
  \item
    We said norms/behaviors are different things, practices are not
    rules
  \item
    Plans are something separate from behavior conforming to the plans
  \item
    Not all social practices generate social rules -- e.g.~shouldn't
    drop the toaster in the bath

    \begin{itemize}
    \tightlist
    \item
      Hart failed to specify what kinds of practices generate rules
    \end{itemize}
  \item
    Plan has to be created with group in mind, accepted by most of
    officials. Thus not all social practices generate shared plans, only
    the ones that involve plan adoption and sharing
  \item
    Unlike Hart, planning theory melds all secondary rules into ``master
    plan'' for the system
  \end{itemize}
\item
  Hume's challenge: how is it that merely by adopting a plan, legal
  authority and obligations can be generated?

  \begin{itemize}
  \tightlist
  \item
    How do we get obligations, rights from social facts alone?
  \item
    Plan could be highly unjust
  \item
    Doesn't generate \emph{moral} obligations, but generates normativity
    that comes from instrumental rationality (rational obligations)
  \item
    How do you get legal obligations out of mere instrumental
    rationality? You don't
  \item
    2 roles that the word ``legal'' plays:

    \begin{enumerate}
    \def\labelenumi{\arabic{enumi}.}
    \tightlist
    \item
      Adjectival interpretation: ``authority'' and ``obligation'' are
      moral concepts,so ``legal authority'' is a moral authority of a
      certain sort, but a legal moral authority -- arises by virtue of
      being part of legal system

      \begin{itemize}
      \tightlist
      \item
        Can't get moral obligations from mere social facts
      \end{itemize}
    \item
      Perspectival interpreation: way of distancing from claims of moral
      obligation/authority

      \begin{itemize}
      \tightlist
      \item
        From the perspective of the law, have moral authority/morally
        obligated to do something
      \item
        From the \emph{legal} POV, you're obligated to do something --
        way of being agnostic towards claims of the law
      \item
        Legal POV is a moral theory according to which the shared plan
        of legal system allocates rights and responsibilities correctly

        \begin{itemize}
        \tightlist
        \item
          This POV says the shared plan is the morally right plan
        \end{itemize}
      \item
        Social facts that generate shared plan also generate a normative
        perspective that sees shared plan as morally appropriate
      \item
        So can generate moral obligations if using the word ``legal''
        perspectivally
      \end{itemize}
    \end{enumerate}
  \end{itemize}
\end{itemize}

\hypertarget{why-im-not-an-inclusive-legal-positivist-part-ii}{%
\subsection{Why I'm not an inclusive legal positivist, part
II}\label{why-im-not-an-inclusive-legal-positivist-part-ii}}

\begin{itemize}
\tightlist
\item
  A plan cuts off deliberation about what you're supposed to do under
  certain circumstances

  \begin{itemize}
  \tightlist
  \item
    Don't balance reasons; point of the plan is to answer the question
    for you
  \end{itemize}
\item
  Simple logic of planning: the existence and content of a plan can't be
  determined by facts whose existence the plan aims to settle

  \begin{itemize}
  \tightlist
  \item
    If you have to deliberate in order to discover the plan, then you
    don't have a plan
  \item
    Inclusive legal positivism violates this: if the point of having law
    is to settle matters about what morality requires so people can
    realize certain goals and values, then legal norms would be useless
    if the way to discover them is to engage in moral reasoning
  \item
    Legal norms without institutional content are like can openers that
    only work when the can is already open
  \item
    Requires you to know what the law is supposed to answer in order to
    know what the law is
  \item
    Exclusive legal positivism doesn't violate this: no danger that the
    process of legal discovery will violate the purpose of having the
    law

    \begin{itemize}
    \tightlist
    \item
      Social facts are determined by empirical observation, not moral
      deliberation
    \item
      Enables law to play plan-like function of answering questions
      about what we ought to do
    \end{itemize}
  \item
    But ELP doesn't rule out possibility that legal norms have social
    foundations that can have moral concepts

    \begin{itemize}
    \tightlist
    \item
      ``Unconscionable contracts'' is a moral concept; says it is
      grossly unfair to hold people to terms of this contract
    \item
      No problem with exclusive legal positivist accepting this. Doesn't
      tell us \emph{when} contract is unconscionable but says that
      unconscionable contracts should not be enforced
    \item
      Judges not engaged in unrestricted moral delibation: judges ask
      whether contract is unconscionable, law still serves as plan
      because it takes some moral issues off the table
    \item
      Deliberation is \emph{channeled} in a certain direction
    \end{itemize}
  \item
    ELP rejects idea that a norm without institutional pedigree can
    still be a law

    \begin{itemize}
    \tightlist
    \item
      E.g.: Contracts that charge 20\% interest rate are unconscionable
    \item
      But no legal institution has decided that 20\% is usurious and
      therefore unsconscionable
    \item
      ILP: since 20\% is unconscionable, any contract that charges 20\%
      is legally unenforceable
    \item
      ELP says the fact that no legal institution has picked that
      interest rate as being unconscionable means that it cannot be
      illegal
    \item
      Once an institution has decided 20\% is unconscionable, then it's
      the law that 20\% contracts are legally unforceable
    \end{itemize}
  \end{itemize}
\end{itemize}

\hypertarget{logic-of-planning}{%
\subsection{Logic of planning}\label{logic-of-planning}}

\begin{itemize}
\tightlist
\item
  If you think laws are plans, you must reject Dworkin's theory
\item
  General logic of planning: when are circumstances complex and
  contentious, alternative forms of social planning increase costs of
  deliberation to the point that we never get anything solved

  \begin{itemize}
  \tightlist
  \item
    So we need a sophisticated institutional form of social planning:
    the law
  \end{itemize}
\item
  Dworkin's constructive interpretation is about discovering content of
  law by engaging in moral deliberation

  \begin{itemize}
  \tightlist
  \item
    Figuring out grounds of law requires figuring out what would make
    legal practice the best it can be -- what fits and justifies legal
    practice
  \item
    If we're thinking about moral fit and justification, this is
    introducing the considerations the law is supposed to take off the
    table
  \item
    Interpretation of any member of system of plans cannot be determined
    by facts any member of that system aims to settle

    \begin{itemize}
    \tightlist
    \item
      Recall the simple logic of planning: existence and content of plan
      can't be determined by facts that plan aims to answer
    \item
      This takes it further: interpretation of any member can't be
      determined by \emph{other} member of the system either
    \end{itemize}
  \end{itemize}
\item
  Dworkin really violates this, even more than ILP does

  \begin{itemize}
  \tightlist
  \item
    ILP permits moral considerations to determine existence and content
    of legal norms even though legal norms aim to settle these moral
    considerations
  \item
    Dworkin \emph{unsettles} questions that have been settled -- have to
    delve into moral philosophy to answer questions about legal
    practice, renders previous decisions by legal institutions on these
    issues moot

    \begin{itemize}
    \tightlist
    \item
      Puts issues back on the table, frustrates ability to guide conduct
      in complex circumstances
    \end{itemize}
  \end{itemize}
\end{itemize}

\hypertarget{meta-interpretation}{%
\section{Meta-interpretation}\label{meta-interpretation}}

\hypertarget{interpretive-methodologies}{%
\subsection{Interpretive
methodologies}\label{interpretive-methodologies}}

\begin{itemize}
\tightlist
\item
  What is legal interpretation? The process of telling you what the law
  is
\item
  The method you use is an \emph{interpretive methodology}: a function
  that takes legal texts and produces the law

  \begin{itemize}
  \tightlist
  \item
    Originalism, textualism, living constitutionalism, intentionalism,
    etc.
  \end{itemize}
\item
  It could be that there's one interpretive methodology valid for every
  legal system

  \begin{itemize}
  \tightlist
  \item
    But we will assume that every legal system has its own interpretive
    methodology -- might need to distinguish federal system, state, etc.
  \end{itemize}
\item
  Interpretive methodology doesn't just take text into account.

  \begin{itemize}
  \tightlist
  \item
    E.g. originalism: take into account original understanding
  \item
    Other events, mental states of those who framed the provisions are
    important
  \end{itemize}
\item
  So, interpretive methodology is more precisely a function that takes
  legal texts, mental states, events, social practices as arguments and
  delivers legal norms
\item
  Legal interpretations/methodology either give you the law or
  \emph{extend} the law

  \begin{itemize}
  \tightlist
  \item
    Ex: prohibition against cruel punishment. Say the death penalty is
    cruel, but no court has ruled this

    \begin{itemize}
    \tightlist
    \item
      ELP: law doesn't prohibit death penalty until court says so
    \item
      Moral fact that death penalty is cruel, so court should extend law
      to have it cover the death penalty
    \item
      So when you decide death penalty is cruel, you're extending the
      law in applying moral reasoning
    \end{itemize}
  \item
    When a court has to decide that the death penalty is cruel and
    therefore prohibited, the court is interpreting the law

    \begin{itemize}
    \tightlist
    \item
      ELP considers this \emph{extending} the law
    \item
      So we don't want to say interpretation is just finding the law
      (like Ronald Dworkin would say)
    \end{itemize}
  \end{itemize}
\item
  Meta-interpretive theory: not a methodology for interpreting legal
  texts, but a methodology for figuring out which interpretive
  methodology is proper

  \begin{itemize}
  \tightlist
  \item
    Dworkin's constructive interpretation is a theory of
    meta-interpretation -- interpretive methodology is the right one
    when it places legal principles in their best light, which is law as
    integrity

    \begin{itemize}
    \tightlist
    \item
      First stage of legal interpretation is meta-interpretation:
      Dworkin thinks you do constructive interpretation here
    \item
      Second stage is applying the interpretive methodology, and for
      Dworkin, meta-interpretation gives you law as integrity, so at
      stage 2, you apply law as integrity to figure out what the law is
    \end{itemize}
  \item
    Dworkin doesn't really distinguish between the 2 stages
  \end{itemize}
\end{itemize}

\hypertarget{the-standard-picture}{%
\subsection{The standard picture}\label{the-standard-picture}}

\begin{itemize}
\tightlist
\item
  Standard picture of legal interpretation: legal content is linguistic
  content, so if you want to know what the law is, figure out what the
  legal texts mean

  \begin{itemize}
  \tightlist
  \item
    Hart: If you want to understand legal texts, explain what it means
    in natural language
  \item
    You get the standard picture when people talk about judges: ``judges
    should stick to the text,'' and if they don't, they will be creating
    law and that's not their role

    \begin{itemize}
    \tightlist
    \item
      If they don't, imposing their own moral and political views on the
      law
    \end{itemize}
  \item
    Hart: open texture and vagueness of natural language means law has
    indeterminacy

    \begin{itemize}
    \tightlist
    \item
      Get judicial creativity because judges have to make new law to
      plug the holes left open by language
    \end{itemize}
  \item
    Standard picture needs to argue why they're choosing the meaning
    they choose, but they often don't
  \item
    Bad at dealing with case law: contribution a case makes to the law
    is just what it says? Often what a court \emph{does} is more
    important than what it says
  \end{itemize}
\item
  Two kinds of meta-interpretive theories: one says look at social facts
  (maybe linguistic meaning, maybe what a court did), or look at moral
  facts too (more of a natural law approach to meta-interpretation)

  \begin{itemize}
  \tightlist
  \item
    Dworkin's theory of constructive interpretation (at least stage one)
    is well-developed natural law theory: thinks right interpretive
    theory is the one that puts legal practice in its morally best light
  \item
    We have yet to see a meta-interpretive theory from a positivist
  \end{itemize}
\item
  Dworkin thinks the reason theoretical disagreements are common is that
  meta-interpretation involves moral reasoning, which is highly
  contested
\end{itemize}

\hypertarget{economy-of-trust}{%
\subsection{Economy of trust}\label{economy-of-trust}}

\begin{itemize}
\tightlist
\item
  Hart says the right interpretation is given by rule of recognition
\item
  Suggest another positivistic theory based on planning theory. Right
  meta-interpretive theory depends on identity of law, and according to
  planning theory, law is a planning system

  \begin{itemize}
  \tightlist
  \item
    So the right way to interpret the law is the right way to interpret
    plans
  \end{itemize}
\item
  We know how to interpret plans. One idea is to interpret plans
  according to their purpose - Flawed because plans are adopted not just
  to achieve a certain purpose, but also because they are trying to deal
  with certain constraints that make decisions on the spot difficult -
  E.g. lack of trust - More distrustful the plan is in the subject of
  the plan, the less discretion they should have in interpreting the
  text
\item
  Expand to the law: think of legal systems as distributions of trust
  and distrust

  \begin{itemize}
  \tightlist
  \item
    If we are more distrustful in legal officials, we leave them less
    discretion to interpret the law in accordance with its purpose
  \item
    Call this the ``economy of trust''
  \item
    Goal of meta-interpretation is to find interpretive methodology that
    best harmonizes with that system's economy of trust
  \item
    Textualists and orginalists say judges shouldn't have a lot of
    discretion because the American legal system is highly distrustful
  \end{itemize}
\item
  Trust is not the only thing that matters in planning. Also level of
  conflict of interest between the planners and the planned

  \begin{itemize}
  \tightlist
  \item
    When we disagree with one another, if we plan, we can still
    coordinate
  \item
    More pluralism means there's less discretion judges should have,
    since plans were generated precisely because there was disagreement
    about deeper issues, and you don't want interpreters to go to deeper
    issues in cases of doubt, because then you're undoing what the plan
    was meant to solve
  \end{itemize}
\item
  Meta-interpreter should figure out the economy of trust of a legal
  system and level of pluralism, and accord the appropriate discretion
  to interpreters
\item
  Disagreement about interpretive methodology can be reconstructed as
  disagreements about level of trust and the level of conflict of values
\item
  These are social facts, so this is a positivistic theory
\end{itemize}
